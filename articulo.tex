%%=====================================================================================
%%
%%       Filename:  articulo.tex
%%
%%    Description:  Artículo
%%
%%        Version:  1.0
%%        Created:  10/03/2018
%%       Revision:  none
%%
%%         Author:  Herbert Arias (dayanqwe123@gmail.com), 
%%   Organization:  
%%      Copyright:  Copyright (c) 2018, Herbert Arias.
%%
%%          Notes:  
%%
%%=====================================================================================

\documentclass[a4paper]{article}

\usepackage[spanish]{babel}
\usepackage[utf8]{inputenc}
\usepackage{hyperref}
\usepackage{amsmath}
\usepackage{graphicx}
\usepackage[colorinlistoftodos]{todonotes}

\title{Un breve recorrido por las tecnologías de Desarrollo Web en el 2018 }

\author{Herbert Brice Arias Silva}

\date{\today}

\begin{document}
\maketitle

\begin{abstract}
   En la actualidad la tecnología tiene un avance vertiginoso y esto genera
   mucho desconcierto al enfrentarse con la decisión de empezar el estudio de
   una de sus ramas, y este avance tiene un cambio mucho mayor en el ámbito de
   la informática ya que comunidades enteras de software así como empresas muy
   grandes del rubro están trabajando en el desarrollo de nuevas y mejoradas
   tecnologías que están reemplazando muy rápido a otras tecnologías
   consideradas nuevas y muy usadas años atrás. En este artículo se realiza un
   breve recopilatorio de esas tecnologías su origen y uso en la Programación
   Web, veremos en líneas generales un panorama de cómo un estudiante de
   primeros ciclos puede abarcar desde el inicio, una carrera profesional
   enfocada a la Programación Web y todo los conocimientos extras que implica
   adquirir a lo largo de su estudio universitario.
\end{abstract}

\section{Introducción}
¿Por qué Programación Web? Es probablemente una pregunta muy acertada y justa.
La respuesta es simple: en la actualidad estamos viviendo una época en la que
la revolución de la informática a llegado al punto de que no hay ámbito de la
industria y de las ciencias que sea ajena al uso de un sistema de información
computarizado a través de software, incluso en nuestra vida diaria tenemos la
influencia de aplicaciones móviles, por ejemplo, que nos ayudan a vivir de una
forma mucho más fácil. La programación Web no está solo vinculada con el
desarrollo de las páginas web y las aplicaciones móviles, sino que se involucra
también a servicios que muchas veces usamos pero no conocemos.

Invito a mis compañeros de carrera a empezar a tomar muy enserio nuestras
posibilidades ya que incluso con pocos conocimientos es posible crear cosas
grandiosas.

Por ejemplo cito a la aplicación Uber, que es una aplicación que ha cambiado
completamente el servicio de transporte en nuestra ciudad, lo estamos viviendo,
y usa tecnologías web que podemos aprender como MySql y PostgreSql, como base
de datos además de lenguajes que utilizan son Javascript, Python, Node.js, Go,
Java, C, C++, Objective C y Swift.

El objetivo principal de este breve artículo no es el de solo informar sino el
de instar a formular proyectos desde ahora ya que el camino de la programación
Web implica adquirir conocimientos, pero más importante aplicar esos
conocimientos y en el camino descubrir qué nuevos conocimientos debemos adquirir.

El objetivo inicial es motivar a mis compañeros de carrera a involucrarse con
el desarrollo Web y que conozcan las herramientas y tecnologías así como
recursos a la que todos podemos acceder ya que Internet es libre, pero que
muchas veces por desconocimiento no logramos encontrar y aprovechar.
Finalmente lograr alcanzar los conocimientos necesarios para una vez egresados
puedan enfrentarse con mucho mayor profesionalismo al desarrollo de proyectos
que tengan impacto social

\section{Bibliografía }

\begin{itemize}
   \item The 2018 Web Developer Roadmap - An ilustrated map to becoming a
      Fronted or Backend Developer with links to courses. [ONLINE] Available
      at: \url{https://codeburst.io/the-2018-web-developer-roadmap-826b1b806e8d}.
      [Accessed 3 Oct 2018]
\end{itemize}

\begin{itemize}
   \item Sergio Lujan Mora (2001). Programación en internet. Cientes web [ONLINE] Available
      at: \url{http://hdl.handle.net/10045/16994}
      [Accessed 3 Oct 2018]
\end{itemize}

\begin{itemize}
   \item ¿Web 2.0? ¿web social? ¿qué es eso? - eLiS e-prints in Library &
      information science [ONLINE] Available at:
      \url{http://eprints.rclis.org/10566/} [Accessed 3 Oct 2018]
\end{itemize}

\begin{itemize}
   \item Web Developer Roadmap 2018 - GitHub, Inc. [ONLINE] Available at:
      \url{https://github.com/kamranahmedse/developer-roadmap} [Accessed 3 Oct 2018]

\end{itemize}

\begin{itemize}
   \item Scott Chacon y Ben Straub (2014). Pro Git Second Edition. [ONLINE] Available at:
      \url{https://git-scm.com/book/en/v2} [Accessed 27 Oct 2018]

\end{document}
