%%=====================================================================================
%%
%%       Filename:  articulo.tex
%%
%%    Description:  Artículo
%%
%%        Version:  1.0
%%        Created:  10/03/2018
%%       Revision:  none
%%
%%         Author:  Herbert Arias (dayanqwe123@gmail.com), 
%%   Organization:  
%%      Copyright:  Copyright (c) 2018, Herbert Arias.
%%
%%          Notes:  
%%
%%=====================================================================================

\documentclass[a4paper]{article}

\usepackage[spanish]{babel}
\usepackage[utf8]{inputenc}
\usepackage{hyperref}
\usepackage{amsmath}
\usepackage{graphicx}
\usepackage[colorinlistoftodos]{todonotes}

\title{Un breve recorrido por las tecnologías de Desarrollo Web en el 2018 }

\author{Herbert Brice Arias Silva}

\date{\today}

\begin{document}
\maketitle

\begin{abstract}
   En la actualidad la tecnología tiene un avance vertiginoso y esto genera
   mucho desconcierto al enfrentarse con la decisión de empezar el estudio de
   una de sus ramas, y este avance tiene un cambio mucho mayor en el ámbito de
   la informática ya que comunidades enteras de software así como empresas muy
   grandes del rubro están trabajando en el desarrollo de nuevas y mejoradas
   tecnologías que están reemplazando muy rápido a otras tecnologías
   consideradas nuevas y muy usadas años atrás. En este artículo se realiza un
   breve recopilatorio de esas tecnologías su origen y uso en la Programación
   Web, veremos en líneas generales un panorama de cómo un estudiante de
   primeros ciclos puede abarcar desde el inicio, una carrera profesional
   enfocada a la Programación Web y todo los conocimientos extras que implica
   adquirir a lo largo de su estudio universitario.
\end{abstract}

\section{Introducción}
Hoy en día casi todo requiere un tipo de programación. Asi pues, ¿qué es?
Programar es básicamente explicarle a tu ordenador que quieres que haga por ti.
Pero citemos lo que piensan sobre esto a grandes programadores que han
revolucionado algun sector con la programación.
Gabe Newell (creador de Valve): "Cuando estás programando le estás enseñando a la cosa posiblemente más
estúpida del universo, un ordenador, a hacer algo".
Mark Zuckerberg (creador de Facebook): "Programar es una de las pocas cosas en el mundo que puedes
hacer cuando estás sentado y simplemente crear algo completamente nuevo desde
cero"
Drew Houston (creador de Dropbox): "Realmente no es muy diferente de tocar un instrumento o
practicar un deporte. Empieza siendo algo muy intimidante, pero terminas por
cogerle el truco."
Chris Bosh (programador de la NBA): "Programar es algo que puede aprenderse. Y sé que puede ser
intimidante, muchas cosas son intimidantes. Pero ya sabes ¿qué no lo es?

La programación es algo absolutamente necesaria en nuestra época, ya que está
en el centro de los mejores productos de la tierra, el software ya se apoderó
del mundo. Saber código hace a cualquier profesión mejor, ya que si se aprendes
programar te dará una capacidad impresionante de cambiar tu profesión. Las
personas más exitosas, la gente que tiene los proyectos más exitosos grandes y
de crecimiento en Internet son aquellos que tienen la intersección de dos
conocimientos y uno de esos conocimientos necesarios es la programación.

En el colegio hemos aprendido cosas muy complejas, y más para ingresar a la
universidad se requiere tener cierto nivel de conocimientos complicados como
por ejemplo en química, balancear una ecuación por Redox o en física el uso de
ecuaciones para interpretar el movimiento parabólico de los cuerpos con masa y
aceleración, pero aprender física o química incluso nos puede acercar en cierto
momento a querer aprender programación ya que aprender sobre los
semiconductores en química o los circuitos, y teoría de transistores en física
nos acerca a la programación porque tienen mucho que ver. Entender los
fundamentos de la programación es mucho más sencillo
que todo eso. Pero ¿por qué muchos sino la mayoría no aprendemos programación
durante el colegio o más crítico aún durante la universidad?, esto tiene que
ver con el Álgebra y cálculo, y es que con eso saltamos a las matemáticas que
casi siempre son útiles para Ingeniería Civil o para la ingeniería Bioquímica
pero no necesariamente para la Ingeniería de Sistemas, Ingeniería de Software o
las ciencias de la computación, por ejemplo nos enseñan límites, nos enseñan
integrales, nos enseñan a calcular el área bajo la curva, y es algo muy
importante, pero es algo muy denso, es como si pasáramos de aprender a conducir
un automovil automático a aprender a conducir un automóvil del la fórmula uno,
y programar deja de ser prioridad en la vida universitaria de muchos futuros
ingenieros.

Saber programar es importante y ya quedó claro el por qué, pero ¿por qué
aprender programación Web? Es probablemente una pregunta muy importante.
Y la respuesta llega de inmediato cuando pensamos en el internet y la Web que
son muy importantes para el uso conectado, además de las computadoras, los
smartphone y muchos otros dispositivos inteligentes que usamos, todos tienen un
componente software que tiene que ver con la programación Web, y es que la
programación Web es amplia y muy interesante. Por ejemplo cito a la aplicación
Uber, que es una aplicación que ha cambiado completamente el servicio de
transporte en nuestra ciudad, lo estamos viviendo, y usa tecnologías web que
podemos aprender como MySql y PostgreSql, como base de datos además de
lenguajes como Javascript, Python, Node.js, Go, Java, C, C++, Objective C y
Swift.

El objetivo principal de este breve artículo no es solo el de informar qué
tecnologías web son a las que podemos apuntar aprender sino el de instar a
formular proyectos desde ahora ya que el camino de la programación Web implica
adquirir conocimientos, pero más importante aplicarlos y en el camino descubrir
qué nuevos conocimientos debemos adquirir.

Inicialmente el artículo busca motivar a mis compañeros de base a involucrarse
con el desarrollo Web y que conozcan las herramientas y tecnologías así como
recursos de aprendizaje a la que todos podemos acceder ya que Internet es
libre, pero muchas veces no conocemos los recursos y no logramos encontrarlos y
aprovecharlos.

\section{Requerimientos necesarios}
Al formarse como programador Web es necesario escoger una ruta, ya que para
dedicarse a esta rama de la informática incluso hay que especializarse. Se
puede elegir entre dos caminos o rutas: Fronted y Backend, pero antes de entrar
en detalle de estas dos ruta es necesario conocer algunas herramientas y
fundamentos que nos ayudarán sin importar cual sea el camino que hayamos
elegido.
\subsection{Git - Sistema de control de versiones}



\newpage
\section{Bibliografía }

\begin{itemize}
   \item The 2018 Web Developer Roadmap - An ilustrated map to becoming a
      Fronted or Backend Developer with links to courses. [ONLINE] Available
      at: \url{https://codeburst.io/the-2018-web-developer-roadmap-826b1b806e8d}.
      [Accessed 28 Oct 2018]
\end{itemize}

\begin{itemize}
   \item Sergio Lujan Mora (2001). Programación en internet. Cientes web [ONLINE] Available
      at: \url{http://hdl.handle.net/10045/16994}
      [Accessed 28 Oct 2018]
\end{itemize}

\begin{itemize}
   \item ¿Web 2.0? ¿web social? ¿qué es eso? - eLiS e-prints in Library and
      information science [ONLINE] Available at:
      \url{http://eprints.rclis.org/10566/} [Accessed 28 Oct 2018]
\end{itemize}

\begin{itemize}
   \item Web Developer Roadmap 2018 - GitHub, Inc. [ONLINE] Available at:
      \url{https://github.com/kamranahmedse/developer-roadmap} [Accessed 28 Oct 2018]
\end{itemize}

\begin{itemize}
   \item Scott Chacon y Ben Straub (2014). Pro Git Second Edition. [ONLINE] Available at:
      \url{https://git-scm.com/book/en/v2} [Accessed 27 Oct 2018]
\end{itemize}

\begin{itemize}
   \item SSH (Secure Shell) Home Page [ONLINE] Available at:
      \url{https://www.ssh.com/ssh/} [Accessed 28 Oct 2018]
\end{itemize}

\begin{itemize}
   \item The HTTPS - Only Standard [ONLINE] Available at:
      \url{https://https.cio.gov/} [Accessed 28 Oct 2018]
\end{itemize}

\begin{itemize}
   \item Hypertext Transfer Protocol - HTTP [ONLINE] Available at:
      \url{https://www.w3.org/Protocols/rfc2616/rfc2616.html} [Accessed 28 Oct 2018]
\end{itemize}

\begin{itemize}
   \item Christopher Negus (2005). Linux Bible Ninth Edition, The comprehensive, tutorial resource.
\end{itemize}

\begin{itemize}
   \item Luis Joyanes Aguilar (2008). Fundamentos de Programación.
\end{itemize}

\begin{itemize}
   \item GitHub Guides [ONLINE] Available at:
      \url{https://guides.github.com/} [Accessed 28 Oct 2018]
\end{itemize}

\begin{itemize}
   \item HTML Tutorial [ONLINE] Available at:
      \url{https://www.w3schools.com/html/} [Accessed 28 Oct 2018]
\end{itemize}

\begin{itemize}
   \item Start learning JavaScrip with our free real time tutorial [ONLINE] Available at:
      \url{https://www.javascript.com/try} [Accessed 28 Oct 2018]
\end{itemize}

\begin{itemize}
   \item The Python Tutorial [ONLINE] Available at:
      \url{https://docs.python.org/3/tutorial/index.html} [Accessed 28 Oct 2018]
\end{itemize}

\begin{itemize}
   \item JavaScript Tutorial [ONLINE] Available at:
      \url{https://www.w3schools.com/js/} [Accessed 28 Oct 2018]
\end{itemize}

\begin{itemize}
   \item Introducción a la programación con Python 3 [ONLINE] Available at:
      \url{http://repositori.uji.es/xmlui/handle/10234/102653?locale-attribute=es} [Accessed 28 Oct 2018]
\end{itemize}


\end{document}
